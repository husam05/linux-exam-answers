\documentclass[a4paper,11pt]{article}
\usepackage[margin=0.75in,top=0.8in,bottom=0.7in]{geometry}
\usepackage{fancyhdr}
\usepackage{amsmath}
\usepackage{xcolor}
\usepackage{tikz}
\usepackage{enumitem}
\usepackage[utf8]{inputenc}
\usepackage{tcolorbox}
\usepackage{booktabs}

% Professional color scheme
\definecolor{primaryblue}{RGB}{0,51,102}
\definecolor{accentblue}{RGB}{0,102,204}
\definecolor{successgreen}{RGB}{0,128,0}
\definecolor{lightgray}{RGB}{240,240,240}

\pagestyle{fancy}
\fancyhf{}
\fancyhead[L]{\textbf{AlMustafa University - Answer Key}}
\fancyhead[R]{\textbf{Linux Systems Administration - Mid-Term Exam}}
\fancyfoot[C]{\small \textcolor{gray}{Page \thepage}}
\renewcommand{\headrulewidth}{1pt}
\renewcommand{\footrulewidth}{0.8pt}

\setlength{\parindent}{0pt}
\setlength{\parskip}{4pt}

\begin{document}

\begin{center}
{\LARGE\textbf{\textcolor{primaryblue}{ANSWER KEY}}}

\vspace{3mm}
{\large Linux Systems Administration - Mid-Term Exam}

\vspace{2mm}
{\normalsize Instructor: Dr. Husam Alkinani}

\vspace{1mm}
{\normalsize Date: November 2, 2025}

\vspace{3mm}
\rule{\textwidth}{1pt}
\end{center}

\vspace{5mm}

% ================= SHEET A =================
\section*{Sheet A - Answer Key}

\subsection*{Question 1: Practical Command Execution (10 points)}

\textbf{Command:} \texttt{ls -l /etc/passwd}

\textbf{Expected Output:}
\begin{tcolorbox}[colback=lightgray,colframe=primaryblue,boxrule=0.5pt]
\ttfamily\small
-rw-r--r-- 1 root root 2847 Oct 15 10:30 /etc/passwd
\end{tcolorbox}

\textbf{Explanation:} This command displays detailed file information for \texttt{/etc/passwd} including:
\begin{itemize}[leftmargin=*,itemsep=2pt]
    \item File permissions: \texttt{-rw-r--r--} (owner: read/write, others: read only)
    \item Owner and group: root
    \item File size: 2847 bytes
    \item Contains user account information (usernames, UIDs, home directories)
\end{itemize}

\subsection*{Question 2: Theoretical Knowledge (10 points)}

\textbf{Correct Answer:} \textcolor{successgreen}{\textbf{a) read, write, execute}}

\textbf{Justification:} In Linux file permissions, "rwx" represents the three basic permission types: read (r), write (w), and execute (x) for files and directories.

\vspace{3mm}
\noindent\rule{\textwidth}{0.5pt}
\vspace{3mm}

% ================= SHEET B =================
\section*{Sheet B - Answer Key}

\subsection*{Question 1: Practical Command Execution (10 points)}

\textbf{Command:} \texttt{whoami}

\textbf{Expected Output:}
\begin{tcolorbox}[colback=lightgray,colframe=primaryblue,boxrule=0.5pt]
\ttfamily\small
student
\end{tcolorbox}

\textbf{Explanation:} The \texttt{whoami} command displays:
\begin{itemize}[leftmargin=*,itemsep=2pt]
    \item The username of the currently logged-in user
    \item Simple and quick way to verify your current user identity
    \item Useful for security verification and scripting
\end{itemize}

\subsection*{Question 2: Theoretical Knowledge (10 points)}

\textbf{Correct Answer:} \textcolor{successgreen}{\textbf{b) Kali Linux}}

\textbf{Justification:} Kali Linux is specifically designed for cybersecurity professionals and penetration testing, with pre-installed security tools and utilities.

\vspace{3mm}
\noindent\rule{\textwidth}{0.5pt}
\vspace{3mm}

% ================= SHEET C =================
\section*{Sheet C - Answer Key}

\subsection*{Question 1: Practical Command Execution (10 points)}

\textbf{Command:} \texttt{pwd}

\textbf{Expected Output:}
\begin{tcolorbox}[colback=lightgray,colframe=primaryblue,boxrule=0.5pt]
\ttfamily\small
/home/student
\end{tcolorbox}

\textbf{Explanation:} The \texttt{pwd} (Print Working Directory) command:
\begin{itemize}[leftmargin=*,itemsep=2pt]
    \item Displays the absolute path of the current working directory
    \item Shows your current location in the filesystem hierarchy
    \item Essential for navigation and file operations
\end{itemize}

\subsection*{Question 2: Theoretical Knowledge (10 points)}

\textbf{Correct Answer:} \textcolor{successgreen}{\textbf{a) Super user do}}

\textbf{Justification:} "sudo" stands for "Super user do" and allows authorized users to execute commands with elevated (root) privileges temporarily.

\newpage

% ================= SHEET D =================
\section*{Sheet D - Answer Key}

\subsection*{Question 1: Practical Command Execution (10 points)}

\textbf{Command:} \texttt{df -h}

\textbf{Expected Output:}
\begin{tcolorbox}[colback=lightgray,colframe=primaryblue,boxrule=0.5pt]
\ttfamily\footnotesize
Filesystem      Size  Used Avail Use\% Mounted on\\
/dev/sda1        50G   20G   28G  42\% /\\
tmpfs           3.9G     0  3.9G   0\% /dev/shm\\
/dev/sda2       100G   45G   50G  48\% /home
\end{tcolorbox}

\textbf{Explanation:} The \texttt{df -h} command displays:
\begin{itemize}[leftmargin=*,itemsep=2pt]
    \item Disk space usage for all mounted filesystems
    \item Human-readable format (GB, MB instead of bytes)
    \item Shows total size, used space, available space, and usage percentage
\end{itemize}

\subsection*{Question 2: Theoretical Knowledge (10 points)}

\textbf{Correct Answer:} \textcolor{successgreen}{\textbf{b) Disk space usage in human-readable format}}

\textbf{Justification:} The \texttt{df -h} command reports filesystem disk space usage in human-readable format (KB, MB, GB) instead of block sizes.

\vspace{3mm}
\noindent\rule{\textwidth}{0.5pt}
\vspace{3mm}

% ================= SHEET E =================
\section*{Sheet E - Answer Key}

\subsection*{Question 1: Practical Command Execution (10 points)}

\textbf{Command:} \texttt{uname -a}

\textbf{Expected Output:}
\begin{tcolorbox}[colback=lightgray,colframe=primaryblue,boxrule=0.5pt]
\ttfamily\footnotesize
Linux ubuntu 5.15.0-56-generic \#62-Ubuntu SMP x86\_64 GNU/Linux
\end{tcolorbox}

\textbf{Explanation:} The \texttt{uname -a} command provides:
\begin{itemize}[leftmargin=*,itemsep=2pt]
    \item Complete system information including kernel name, version, and architecture
    \item Operating system type (Linux)
    \item Machine hardware name and processor type
\end{itemize}

\subsection*{Question 2: Theoretical Knowledge (10 points)}

\textbf{Correct Answer:} \textcolor{successgreen}{\textbf{b) Complete system information including kernel version}}

\textbf{Justification:} The \texttt{uname -a} command displays all available system information including kernel version, hostname, and architecture.

\vspace{3mm}
\noindent\rule{\textwidth}{0.5pt}
\vspace{3mm}

% ================= SHEET F =================
\section*{Sheet F - Answer Key}

\subsection*{Question 1: Practical Command Execution (10 points)}

\textbf{Command:} \texttt{top}

\textbf{Expected Output:}
\begin{tcolorbox}[colback=lightgray,colframe=primaryblue,boxrule=0.5pt]
\ttfamily\footnotesize
Tasks: 195 total,   1 running, 194 sleeping\\
\%Cpu(s):  2.3 us,  1.0 sy,  0.0 ni, 96.7 id\\
MiB Mem :   7892.5 total,   3421.2 free\\
PID USER      PR  NI    VIRT    RES  \%CPU  \%MEM COMMAND
\end{tcolorbox}

\textbf{Explanation:} The \texttt{top} command displays:
\begin{itemize}[leftmargin=*,itemsep=2pt]
    \item Real-time system process monitoring
    \item CPU and memory usage statistics
    \item Running processes with resource consumption details
\end{itemize}

\subsection*{Question 2: Theoretical Knowledge (10 points)}

\textbf{Correct Answer:} \textcolor{successgreen}{\textbf{b) Monitor real-time system processes and resource usage}}

\textbf{Justification:} The \texttt{top} command provides dynamic real-time view of running system processes, CPU usage, memory consumption, and system load.

\vspace{5mm}

\begin{center}
\rule{0.5\textwidth}{0.5pt}

\textbf{Grading Rubric}

\begin{tabular}{ll}
\toprule
\textbf{Component} & \textbf{Points} \\
\midrule
Question 1 - Command Output & 5 points \\
Question 1 - Explanation & 5 points \\
Question 2 - Correct Answer & 5 points \\
Question 2 - Justification & 5 points \\
\midrule
\textbf{Total per Sheet} & \textbf{20 points} \\
\bottomrule
\end{tabular}
\end{center}

\end{document}
